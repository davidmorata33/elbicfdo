\subsection{Rotation}
\paragraph{}An important consideration for this project is that the rotor is spinning at a high speed and this condition has to be somehow communicated to \textit{OpenFOAM}. To work with this condition the Multiple Frame Reference (MFR) method has to be used. The method itself is based on adding a 'source' to the momentum equation in a previously defined zone named 'rotor'. How to do that has been shown in the previous section.

\paragraph{}The \textbf{constant/MRFProperties} file has to be modified with the appropriate parameters for the case. Since the axis of \textit{OpenFOAM} are based on the right hand rule, the rotation axis must be defined as follows: \textbf{CAMBIAR PER ARA!} $(1,0,0)$. From information that can be found only, it has been chosen 5000 $rpm$ as the angular velocity of the rotor. It should be noted that the units used must be those of the International System, so the 5000 $rpm$ must be converted to $rad/s$. Furthermore, an origin point has to be indicated in the axis of rotation; in this case, the origin point is as follows: $(0 1.1675354 1.15542633)$. 

\paragraph{}Finally, it has to be indicated wheter some geometries are rotating or not. Thus, the next parameter has to be modified:

\begin{footnotesize}
\begin{verbatim}
    nonRotatingPatches (
	NacelleStator.stl
	);
\end{verbatim}
\end{footnotesize}

\paragraph{}So, the file \textbf{constant/MRFProperties} for the current case is the following:

\begin{footnotesize}
\begin{verbatim}
/*--------------------------------*- C++ -*----------------------------------*\
| =========                 |                                                 |
| \\      /  F ield         | OpenFOAM: The Open Source CFD Toolbox           |
|  \\    /   O peration     | Version:  4.0                                   |
|   \\  /    A nd           | Web:      www.OpenFOAM.org                      |
|    \\/     M anipulation  |                                                 |
\*---------------------------------------------------------------------------*/
FoamFile
{
    version     2.0;
    format      ascii;
    class       dictionary;
    location    "constant";
    object      MRFProperties;
}
// * * * * * * * * * * * * * * * * * * * * * * * * * * * * * * * * * * * * * //

MRF1
{
    cellZone    rotor;
    active      yes;

    // Fixed patches (by default they 'move' with the MRF zone)
    nonRotatingPatches (
	NacelleStator.stl
	);

    origin    (0 1.1675354 1.15542633);
    axis      (1 0 0);
    omega     523.5987756;
}

// ************************************************************************* //
\end{verbatim}
\end{footnotesize}

\paragraph{}It can clearly be seen that the modification of any parameter is pretty straigh-forward in the file since there is no possible confusion with the names.