\subsection{Control}
\paragraph{}
First of all, the initial and final times of the simulation have to be defined. In our case, the \textit{startTime} is 0 s and the \textit{endTime} is 20 s. After this, the time step fot he simulation have to be defined. We are interested on simulate each second of the simulation but we want that the programm provides us the simulation data each 5 seconds, therefore the \textit{deltaT} is 1 s and the \textit{writeInterval} is 5 s.

\paragraph{}
To change all these parameters, we have to modify the file \textbf{system/controlDict}.


\begin{footnotesize}
\begin{verbatim}
/*--------------------------------*- C++ -*----------------------------------*\
| =========                 |                                                 |
| \\      /  F ield         | OpenFOAM: The Open Source CFD Toolbox           |
|  \\    /   O peration     | Version:  4.0                                   |
|   \\  /    A nd           | Web:      www.OpenFOAM.org                      |
|    \\/     M anipulation  |                                                 |
\*---------------------------------------------------------------------------*/
FoamFile
{
    version     2.0;
    format      ascii;
    class       dictionary;
    object      controlDict;
}
// * * * * * * * * * * * * * * * * * * * * * * * * * * * * * * * * * * * * * //

application     simpleFoam;

startFrom       latestTime;

startTime       0;

stopAt          endTime;

endTime         20;

deltaT          1;

writeControl    timeStep;

writeInterval   5;

purgeWrite      0;

writeFormat     binary;

writePrecision  6;

writeCompression uncompressed;

timeFormat      general;

timePrecision   6;

runTimeModifiable true;

/*functions
{
    #include "streamLines"
    #include "wallBoundedStreamLines"
    #include "cuttingPlane"
    #include "forceCoeffs"
}

*/
// ************************************************************************* //

\end{verbatim}
\end{footnotesize}