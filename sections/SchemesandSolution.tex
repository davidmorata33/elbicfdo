\subsection{fvSchemes and fvSolution}
\paragraph{}
The fvSchemes dictionary contains the numerical schemes that they are used in the simulation. In the hypotheses section we have considered that the flow is transient To impose this condition we must modify the \texttt{system/fvSchemes} file, the section of \textit{ddtSchemes}, which is the temporal integration scheme. We must write Euler. Finally, the file that has been used to run the simulation is presented below.

\begin{footnotesize}
\begin{verbatim}
/*--------------------------------*- C++ -*----------------------------------*\
| =========                 |                                                 |
| \\      /  F ield         | OpenFOAM: The Open Source CFD Toolbox           |
|  \\    /   O peration     | Version:  4.x                                   |
|   \\  /    A nd           | Web:      www.OpenFOAM.org                      |
|    \\/     M anipulation  |                                                 |
\*---------------------------------------------------------------------------*/
FoamFile
{
    version     2.0;
    format      ascii;
    class       dictionary;
    location    "system";
    object      fvSchemes;
}
// * * * * * * * * * * * * * * * * * * * * * * * * * * * * * * * * * * * * * //

ddtSchemes
{
    default         Euler;
}

gradSchemes
{
    default         Gauss linear;
    grad(p)         Gauss linear;
    grad(U)         cellLimited Gauss linear 1;
}

divSchemes
{
    default         none;
//    div(phi,U)      Gauss upwind;
    div(phi,U)      Gauss linearUpwind grad(U);
    div((nuEff*dev2(T(grad(U))))) Gauss linear;
}

laplacianSchemes
{
    default         Gauss linear limited corrected 0.33;
}

interpolationSchemes
{
    default         linear;
}

snGradSchemes
{
    default         limited corrected 0.33;
}


// ************************************************************************* //

\end{verbatim}
\end{footnotesize}


\paragraph{}
The \texttt{system/fvSolution} file contains the resolution methods and the tolerances for each equation, but in our case it has not been modified. So, the file fvSolution for the current case is the following:


\begin{footnotesize}
\begin{verbatim}
/*--------------------------------*- C++ -*----------------------------------*\
| =========                 |                                                 |
| \\      /  F ield         | OpenFOAM: The Open Source CFD Toolbox           |
|  \\    /   O peration     | Version:  4.x                                   |
|   \\  /    A nd           | Web:      www.OpenFOAM.org                      |
|    \\/     M anipulation  |                                                 |
\*---------------------------------------------------------------------------*/
FoamFile
{
    version     2.0;
    format      ascii;
    class       dictionary;
    object      fvSolution;
}
// * * * * * * * * * * * * * * * * * * * * * * * * * * * * * * * * * * * * * //

solvers
{
    pcorr
    {
        solver          GAMG;
        tolerance       1e-2;
        relTol          0;
        smoother        DICGaussSeidel;
        cacheAgglomeration no;
        maxIter         50;
    }

    p
    {
        $pcorr;
        tolerance       1e-5;
        relTol          0.01;
    }

    pFinal
    {
        $p;
        tolerance       1e-6;
        relTol          0;
    }

    U 
    {
        solver          smoothSolver;
        smoother        symGaussSeidel;
        tolerance       1e-6;
        relTol          0.1;
    }

    UFinal
    {
        solver          smoothSolver;
        smoother        symGaussSeidel;
        tolerance       1e-6;
        relTol          0;
    }
}

PIMPLE
{
    correctPhi          no;
    nOuterCorrectors    3;
    nCorrectors         1;
    nNonOrthogonalCorrectors 0;
}

relaxationFactors
{
    U       0.5;
    UFinal  1;
}

cache
{
    grad(U);
}

// ************************************************************************* //

\end{verbatim}
\end{footnotesize}
