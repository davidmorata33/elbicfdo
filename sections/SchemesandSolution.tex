\subsection{fvSchemes and fvSolution}
\paragraph{}
The fvSchemes dictionary contains the numerical schemes that they are used in the simulation and the fvSolution contains the resolution methods and the tolerances for each equation.

\paragraph{}
In the hypotheses section we have considered that the flow is stationary. To impose this condition we must modify the \textbf{system/fvSchemes} file, the section of In \textit{ ddtSchemes}, which is the temporal integration scheme. We must write steadyState. The fvSolution dictionary has not been modified.

\paragraph{}
The


\begin{footnotesize}
\begin{verbatim}
/*--------------------------------*- C++ -*----------------------------------*\
| =========                 |                                                 |
| \\      /  F ield         | OpenFOAM: The Open Source CFD Toolbox           |
|  \\    /   O peration     | Version:  4.0                                   |
|   \\  /    A nd           | Web:      www.OpenFOAM.org                      |
|    \\/     M anipulation  |                                                 |
\*---------------------------------------------------------------------------*/
FoamFile
{
    version     2.0;
    format      ascii;
    class       dictionary;
    object      fvSchemes;
}
// * * * * * * * * * * * * * * * * * * * * * * * * * * * * * * * * * * * * * //

ddtSchemes
{
    default         steadyState;
}

gradSchemes
{
    default         Gauss linear;
    grad(U)         cellLimited Gauss linear 1;
}

divSchemes
{
    default         none;
    div(phi,U)      bounded Gauss linearUpwindV grad(U);
    div(phi,k)      bounded Gauss upwind;
    div(phi,omega)  bounded Gauss upwind;
    div((nuEff*dev2(T(grad(U))))) Gauss linear;
}

laplacianSchemes
{
    default         Gauss linear corrected;
}

interpolationSchemes
{
    default         linear;
}

snGradSchemes
{
    default         corrected;
}

wallDist
{
    method meshWave;
}


// ************************************************************************* //

\end{verbatim}
\end{footnotesize}

\begin{footnotesize}
\begin{verbatim}
/*--------------------------------*- C++ -*----------------------------------*\
| =========                 |                                                 |
| \\      /  F ield         | OpenFOAM: The Open Source CFD Toolbox           |
|  \\    /   O peration     | Version:  4.0                                   |
|   \\  /    A nd           | Web:      www.OpenFOAM.org                      |
|    \\/     M anipulation  |                                                 |
\*---------------------------------------------------------------------------*/
FoamFile
{
    version     2.0;
    format      ascii;
    class       dictionary;
    object      fvSolution;
}
// * * * * * * * * * * * * * * * * * * * * * * * * * * * * * * * * * * * * * //

solvers
{
    p
    {
        solver           GAMG;
        tolerance        1e-7;
        relTol           0.01;
        smoother         GaussSeidel;
    }

    Phi
    {
        $p;
    }

    U
    {
        solver           smoothSolver;
        smoother         GaussSeidel;
        tolerance        1e-8;
        relTol           0.1;
        nSweeps          1;
    }

    k
    {
        solver           smoothSolver;
        smoother         GaussSeidel;
        tolerance        1e-8;
        relTol           0.1;
        nSweeps          1;
    }

    omega
    {
        solver           smoothSolver;
        smoother         GaussSeidel;
        tolerance        1e-8;
        relTol           0.1;
        nSweeps          1;
    }
}

SIMPLE
{
    nNonOrthogonalCorrectors 0;
    consistent yes;
}

potentialFlow
{
    nNonOrthogonalCorrectors 10;
}

relaxationFactors
{
    equations
    {
        U               0.9;
        k               0.7;
        omega           0.7;
    }
}

cache
{
    grad(U);
}

// ************************************************************************* //

\end{verbatim}
\end{footnotesize}
