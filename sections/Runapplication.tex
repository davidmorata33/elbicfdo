\subsection{Commands to run the simulation}

\paragraph{}The first thing that must be done when the \textit{blockMesh} has been modified is to type the following lines in a terminal in the folder where the case is located. As it can be seen, it will save the log on a new file named blockMesh.log.

\begin{center}
\texttt{blockMesh >blockMesh.log}
\end{center}

\paragraph{}When the \textit{blockMesh} has been generated, the next thing to do is the refinement of the mesh. So, the \textit{snappyHexMesh} utility will be used, provided that all the parameters in \textit{snappyHexMeshDict} have been modified. In the terminal, the following command has to be typed:

\begin{center}
\texttt{foamJob decomposePar \\
foamJob surfaceFeatureExtract \\
foamJob -s -p snappyHexMesh -overwrite}
\end{center}

\paragraph{}The previous command will make the mesh refinement to run in parallel (notice the \textit{decomposePar} and the \textit{-p}). Additionally, a log will be created when running the command that will contain all the information about the mesh refinement.

\paragraph{}In order to run the desired simulation, a patch with the AMI surface has to be made. Thus, the following command has to be typed. It will also create a \textit{log} file named \textit{createPatch.log}.

\begin{center}
\texttt{createPatch -overwrite > createPatch.log}
\end{center}

\paragraph{}Next, the simulation can be run.

\begin{center}
\texttt{cp -r 0.orig 0\\
foamJob PimpleDyMFoam\\
mv log PimpleDyMFoam.log}
\end{center}

\paragraph{}And finally, given that we were working in parallel, the case has to be reconstructed to show the results.

\begin{center}
\texttt{reconstructParMesh -constant\\
reconstructPar}
\end{center}

\paragraph{}There has also been added into the folder the \texttt{Allrun} file, which will run the case automatically.

\paragraph{}To view the results, typing \texttt{paraFoam} in the \texttt{terminal} window should be enough.