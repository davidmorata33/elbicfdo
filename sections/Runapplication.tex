\subsection{Commands to run the simulation}
\paragraph{}The first thing to perform the simulation is to copy the tutorial folder case into our current directory. To do so, open a terminal window and type the following command:

\begin{center}
\texttt{cp -r /opt/openfoam4/tutorials/incompressible/\\ pimpleDyMFoam/mixerVesselAMI2D/* .
}
\end{center}
\paragraph{}Then we have to modify all the parameters explained during this report. The first thing that must be done when the \textit{blockMeshDict} has been modified is to type the following lines in a terminal. As it can be seen, it will save the log on a new file named blockMesh.log.

\begin{center}
\texttt{blockMesh > blockMesh.log}
\end{center}

\paragraph{}When the \textit{blockMesh} has been generated, the next thing to do is the refinement of the mesh. So, the \textit{snappyHexMesh} utility will be used, provided that all the parameters in \texttt{system/snappyHexMeshDict} have been modified. In the terminal, the following command has to be typed:

\begin{center}
\texttt{foamJob decomposePar \\
foamJob -p surfaceFeatureExtract \\
foamJob -s -p snappyHexMesh -overwrite}
\end{center}

\paragraph{}The previous command will make the mesh refinement to run in parallel (notice the \textit{decomposePar} and the \textit{-p}). Additionally, a log will be created when running the command that will contain all the information about the mesh refinement. The case can be runned without using parallel computing by typing:

\begin{center}
\texttt{surfaceFeatureExtract > surfaceFeatureExtract \\
foamJob -s snappyHexMesh -overwrite\\
mv log snappyHexMesh.log }
\end{center}

\paragraph{}In order to run the desired simulation, a patch with the AMI surface has to be made. Thus, the following command has to be typed. It will also create a \textit{log} file named \textit{createPatch.log}.

\begin{center}
\texttt{createPatch -overwrite > createPatch.log}
\end{center}

\paragraph{}Next, the simulation can be run.

\begin{center}
\texttt{cp -r 0.orig 0\\
foamJob -s PimpleDyMFoam\\
mv log PimpleDyMFoam.log}
\end{center}

\paragraph{}And finally, if we were working in parallel (in that case a -p must be added to the pimpleDyMFoam command as we did in the previous steps), the case has to be reconstructed to show the results.

\begin{center}
\texttt{reconstructParMesh -constant\\
reconstructPar}
\end{center}

\paragraph{}There has also been added into the folder the \texttt{Allrun} file, which will run the case automatically.

\paragraph{}To view the results, typing \texttt{paraFoam} in the \texttt{terminal} window should be enough. The .zip folder of the case delivered along with this report contains an Allrun and an Allclean scripts to automatize all these commands.