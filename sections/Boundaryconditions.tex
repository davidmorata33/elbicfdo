\subsection{Boundary conditions}


\subsubsection{Velocity}

\paragraph{}Some initial conditions and values have to be defined in order to run the simulation. The initial velocity field and pressure are especially important since these inputs are the starting values of the simulation. So, it is clear that the values at the \textit{inlet}, which was defined within the previous sections, have to be defined. Additionally, the values at the \textit{outlet} of the geometry have to be defined as well, but they can be, unlike the \textit{inlet}, not fixed values. Thus, the following modifications have to be made in the  \textit{0.orig/U}.

\begin{footnotesize}
\begin{verbatim}
/*--------------------------------*- C++ -*----------------------------------*\
| =========                 |                                                 |
| \\      /  F ield         | OpenFOAM: The Open Source CFD Toolbox           |
|  \\    /   O peration     | Version:  4.0                                   |
|   \\  /    A nd           | Web:      www.OpenFOAM.org                      |
|    \\/     M anipulation  |                                                 |
\*---------------------------------------------------------------------------*/
FoamFile
{
    version     2.0;
    format      binary;
    class       volVectorField;
    location    "0";
    object      U;
}
// * * * * * * * * * * * * * * * * * * * * * * * * * * * * * * * * * * * * * //

dimensions      [0 1 -1 0 0 0 0];

internalField   uniform (0 0 0);

boundaryField
{
    inlet
    {
        type            fixedValue;
        value           uniform (200 0 0);
    }
    outlet
    {
        type            inletOutlet;
        inletValue      uniform (0 0 0);
        value           uniform (0 0 0);
    }
    LPSpool
    {
        type            movingWallVelocity;
        value           uniform (0 0 0);
    }
    NacelleStator
    {
        type            noSlip;
    }
    AMI1
    {
        type            cyclicAMI;
        value           uniform (0 0 0);
    }
    AMI2
    {
        type            cyclicAMI;
        value           uniform (0 0 0);
    }
}


// ************************************************************************* //
\end{verbatim}
\end{footnotesize}

\paragraph{}As it can be seen above, the value of the velocity at the inlet has been set to 200 $m/s$ (about 720 $km/h$). The \textit{inletOutlet} condition has been set to the outlet (this is: the flow is restricted to move always forwards and reverse flow cannot appear at this region). The NacelleStator file has the noSlip condition given that the fluid is supposed to be Newtonian. Finally, the initial velocity field for the rotational parts AMI1 and AMI2 have been set to 0 (this is: the flow is not moving at the beginning of the simulation).

\subsubsection{Pressure}

\paragraph{}The initial conditions for the pressure have to be evaluated as well in the \textit{0.orig/p} file.

\begin{footnotesize}
\begin{verbatim}
/*--------------------------------*- C++ -*----------------------------------*\
| =========                 |                                                 |
| \\      /  F ield         | OpenFOAM: The Open Source CFD Toolbox           |
|  \\    /   O peration     | Version:  4.0                                   |
|   \\  /    A nd           | Web:      www.OpenFOAM.org                      |
|    \\/     M anipulation  |                                                 |
\*---------------------------------------------------------------------------*/
FoamFile
{
    version     2.0;
    format      binary;
    class       volScalarField;
    location    "0";
    object      p;
}
// * * * * * * * * * * * * * * * * * * * * * * * * * * * * * * * * * * * * * //

dimensions      [0 2 -2 0 0 0 0];

internalField   uniform 0;

boundaryField
{
    inlet
    {
        type            zeroGradient;
    }
    outlet
    {
        type            fixedValue;
        value           uniform 0;
    }
    LPSpool
    {
        type            zeroGradient;
    }
    NacelleStator
    {
        type            zeroGradient;
    }
    AMI1
    {
        type            cyclicAMI;
        value           uniform 0;
    }
    AMI2
    {
        type            cyclicAMI;
        value           uniform 0;
    }
}


// ************************************************************************* //
\end{verbatim}
\end{footnotesize}

\paragraph{}The \textit{zeroGradient} condition simply extrapolates the quantity to the current surface from the nearest cell value meaning that the gradient of the quantity is equal to zero in the direction perpendicular to the partch. This condition will not fix a value for the pressure, it will just let the pressure be the same as the calculated using the Navier Stokes equations (provided that the velocity field is known).

\paragraph{}Finally, the boundary conditions for the \textit{AMI1} and \textit{AMI2} should not be modified since they are specifically set for a geometry that is rotating.